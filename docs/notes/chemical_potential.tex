\documentclass[reprint,onecolumn,groupedaddress,amsmath,amssymb]{revtex4-2}

\usepackage[]{hyperref}
\usepackage{graphicx}
\usepackage{enumitem}
\usepackage{xr}

\hypersetup{
  colorlinks   = true,
  urlcolor     = blue,
  linkcolor    = blue,
  citecolor   = red
}

\begin{document}

\title{Mapping between standard chemical potentials and exchange chemical potentials / osmotic pressure}

\author{Yicheng Qiang}
\affiliation{Max Planck Institute for Dynamics and Self-Organization, Am Fa{\ss}berg 17, 37077 G{\"o}ttingen, Germany}

\maketitle

For a general free energy density of $N_\mathrm{c}$ components $f(\{\phi_i\}_{i=1}^{N_\mathrm{c}})$, with the volume fractions defined as
\begin{align}
    \phi_i = \nu_i  N_i / V\;, \quad \quad \quad V = \sum_j^{N_\mathrm{c}} \nu_j N_j + V_0 \;.
\end{align}
The total free energy is $F = V f$.
Then the (original) chemical potentials are
\begin{align}
    \mu_i 
    &= \frac{\partial F}{\partial N_i}
    = \nu_i f + V \frac{\partial f}{\partial N_i}
    = \nu_i f + V \sum_{j=1}^{N_\mathrm{c}} \frac{\partial \phi_j}{\partial N_i} \frac{\partial f}{\partial \phi_j}
    = \nu_i f + V \sum_{j=1}^{N_\mathrm{c}} \frac{\nu_j \delta_{ij} V - \nu_j N_j \nu_i}{V^2} \frac{\partial f}{\partial \phi_j} \notag \\
    &= \nu_i \left( f - \sum_j \phi_j \frac{\partial f}{\partial \phi_j} + \frac{\partial f}{\partial \phi_i} \right) \;.
\end{align}
Since the dimensionless free energy density is defined by $f = \tilde{f} k_\mathrm{B} T / \nu$, while $\nu$ is a reference molecule volume, the dimensionless chemical potentials become
\begin{align}
    \tilde{\mu}_i = l_i \left(\tilde{f} - \sum_j \phi_j \frac{\partial \tilde{f}}{\partial \phi_j} + \frac{\partial \tilde{f}}{\partial \phi_i} \right) \;,
\end{align} 
where $l_i = \nu_i / \nu$ is the relative molecule volumes.
Then the exchange chemical potentials with respect to component $s$ is simply $\tilde{\mu}_i - \tilde{\mu}_s$.
The osmotic pressure $P_i$, defined by the balance $P_i \nu_i + \partial F/\partial N_i = 0$, reads
\begin{align}
    P_i = - \mu_i / \nu_i, 
\end{align} 
while the dimensionless version $\tilde{P}_i$, defined by $P_i = \tilde{P}_i k_\mathrm{B} T / \nu$, reads
\begin{align}
    \tilde{P}_i = - \tilde{\mu}_i / l_i \;.
\end{align}

To avoid include explicit relative volumes, and keep the definition of original chemical potential and the exchange chemical potential identical, we let $\tilde{\mu}_i$ absorb a prefactor of $1/l_i$, 
\begin{subequations}
    \begin{align}
        \tilde{\mu}_i &=  \tilde{f} - \sum_j \phi_j \frac{\partial \tilde{f}}{\partial \phi_j} + \frac{\partial \tilde{f}}{\partial \phi_i} \\
        \tilde{P}_i &= - \tilde{\mu}_i \;.
    \end{align}
\end{subequations} 
In the package, we use the dimensionless version of the quantities, and drop the tilde for lighter symbols.

\end{document}